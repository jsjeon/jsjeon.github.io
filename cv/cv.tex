% resume.tex
%
% (c) 2002 Matthew Boedicker <mboedick@mboedick.org> (original author) http://mboedick.org
% (c) 2003 David J. Grant <dgrant@ieee.org> http://www.davidgrant.ca
% (c) 2007 Todd C. Miller <Todd.Miller@courtesan.com> http://www.courtesan.com/todd
% (c) 2009 Derek R. Hildreth <derek@derekhildreth.com> http://www.derekhildreth.com 
%This work is licensed under the Creative Commons Attribution-NonCommercial-ShareAlike License. To view a copy of this license, visit http://creativecommons.org/licenses/by-nc-sa/1.0/ or send a letter to Creative Commons, 559 Nathan Abbott Way, Stanford, California 94305, USA.

% GENERAL NOTE:  There may be some notes specific to myself.  If you're only interested in my LaTeX source or it doesn't make sense, please disregard it.

\documentclass[letterpaper,11pt]{article}

%-----------------------------------------------------------
\usepackage{latexsym}
\usepackage[empty]{fullpage}
\usepackage[usenames,dvipsnames]{color}
\usepackage{verbatim}
\usepackage[pdftex, pdftitle={Resume for Jinseong Jeon - \today}]{hyperref} \hypersetup{
    colorlinks,%
    citecolor=black,%
    filecolor=black,%
    linkcolor=black,%
    urlcolor=black    % can put red here to visualize the links
}
\urlstyle{same}
\definecolor{mygrey}{gray}{.85}
\definecolor{mygreylink}{gray}{.30}
\textheight=9.0in
\raggedbottom
\raggedright
\setlength{\tabcolsep}{0in}
\usepackage{fancyhdr}
\rfoot{}
\cfoot{}
\lfoot{} %\lfoot{\small Last Updated: \today}
\pagestyle{fancy}
\renewcommand{\headrulewidth}{0.0pt}
\renewcommand{\footrulewidth}{0.0pt}
% Adjust margins
\addtolength{\oddsidemargin}{-0.375in}
\addtolength{\evensidemargin}{0.375in}
\addtolength{\textwidth}{0.5in}
\addtolength{\topmargin}{-.375in}
\addtolength{\textheight}{0.75in}

%-----------------------------------------------------------
%Custom commands
\newcommand{\resitem}[1]{\item #1 \vspace{-2pt}}
\newcommand{\resheading}[1]{{\large \colorbox{mygrey}{\begin{minipage}{\textwidth}{\textbf{#1 \vphantom{p\^{E}}}}\end{minipage}}}}
\newcommand{\ressubheading}[4]{
\begin{tabular*}{6.5in}{l@{\extracolsep{\fill}}r}
    \textbf{#1} & #2 \\
    \textit{#3} & \textit{#4} \\
\end{tabular*}\vspace{-6pt}}
%-----------------------------------------------------------

%-----------------------------------------------------------
%General Resume Tips
%   No periods!  Technically, nothing in this document is a full sentence.
%   Use parallelism by ending key words with the same thing,  i.e. "Coordinated; Designed; Communicated".
%   More tips on bottom of this LaTeX document.
%-----------------------------------------------------------

\usepackage{natbib}
\renewcommand{\bibfont}{\small}

\usepackage{multibib}
\newcites{art}{Journal Articles}
\newcites{conf}{Conference/Workshop Proceedings}
\newcites{post}{Posters}
\newcites{tr}{Technical Reports}

\begin{document}

\newcommand{\mywebheader}{
\begin{tabular*}{7in}{l@{\extracolsep{\fill}}r}
  \textbf{{\LARGE Jinseong Jeon}}
& \href{mailto:jsjeon@cs.umd.edu}{jsjeon@cs.umd.edu} \hspace{1mm}$\vert$\hspace{1mm} 425-626-9074
\\
  {\small 787 6th St S, Kirkland, WA 98033}
& \href{https://jsjeon.github.io/}{\url{https://jsjeon.github.io/}}
\\
\end{tabular*}
\\
\vspace{0.1in}}

% CHANGE HEADER SOURCE HERE
\mywebheader

%%%%%%%%%%%%%%%%%%%%%%
\resheading{Education}
  \begin{itemize}
    \item
      \ressubheading{\href{https://www.cs.umd.edu}{University of Maryland, College Park}}{College Park, MD}{{Ph.D. in Computer Science}}{Feb 2016}
    \item
      \ressubheading{\href{https://cs.kaist.ac.kr}{KAIST}}{Daejeon, Korea}{{M.S. in Computer Science}}{Feb 2007}
\begin{comment}
Outstanding Master's Thesis Award from Dept. of Computer Science, KAIST
\end{comment}
    \item
      \ressubheading{\href{https://cs.kaist.ac.kr}{KAIST}}{Daejeon, Korea}{{B.S. in Computer Science}}{Feb 2005}
\begin{comment}
Graduated with High Honor (\emph{Magna Cum Laude})
\end{comment}
  \end{itemize} % End Education list

%%%%%%%%%%%%%%%%%%%%%%
\resheading{Experience}
  \begin{itemize}
    \item
      \ressubheading{{Google Inc.}}{Kirkland, WA}{Software Engineer}{Feb 2016 -- Present}
        {
\small
        \begin{itemize}
\item\emph{\href{https://developer.android.com/studio/preview/index.html}{Android Compiler Toolchain}}:
\href{https://r8.googlesource.com/r8}{D8 dexer and R8 shrinker}.

\item\emph{\href{https://cloud.google.com/compute/}{Google Compute Engine}}:
\href{https://cloudplatform.googleblog.com/2017/09/committed-use-discounts-for-Google-Compute-Engine-now-generally-available.html}{Committed use discounts}.
        \end{itemize}
        }
    \item
      \ressubheading{{University of Maryland, College Park}}{College Park, MD}{Research Assistant}{Jun 2011 -- Feb 2016}
        {
\small
        \begin{itemize}
\item\emph{\textsc{Pasket}: Synthesizing Framework Models for Symbolic Execution}~\cite{FMSD17, icse16, fse15, cav15, popl14, jsk-tr}:
Researched and developed scalable synthesis of models for object-oriented,
event-driven frameworks, such as Java Swing and Android,
that enable other static analysis tools
to analyze real-world apps effectively and efficiently.

\item\emph{SymDroid: Symbolic Execution for Dalvik Bytecode}~\cite{esorics15, esorics15-tr, symdroid}:
Developed a symbolic execution engine for Android apps,
which explores apps' possible behaviors and
discovers under what circumstance apps may disclose which sensitive user data.

\item\emph{Brahmastra: Driving Apps to Test the Security of Third-Party Components}~\cite{security14}:
Designed and implemented a static analysis that extracts program execution paths
from app's entry points to method calls of interest, e.g., authorization APIs
in Facebook SDK, which are in turn used to steer apps so as to test
security properties of third-party components in a large scale.

\item\emph{\href{https://github.com/plum-umd/troyd}{Troyd: Integration Testing Framework for Android}}~\cite{troyd}:
Developed a script-based testing framework for Android
that allows testers to run Android apps via command-line interface;
record testing scenarios; and replay recorded scenarios as regression tests.

\item\emph{\href{https://github.com/plum-umd/redexer}{Redexer: Dalvik Bytecode Instrumentation Framework}}~\cite{spsm12, drandroid, acplib}:
Developed a general-purpose bytecode rewriting framework for Android,
which is composed of a rich set of utilities that let programmers
parse, manipulate, and generate Dalvik bytecode from scratch.
        \end{itemize}
        }
    \item
      \ressubheading{{Google Inc.}}{Mountain View, CA}{Software Engineering Intern}{May -- Aug 2015}
        {
\small
        \begin{itemize}
\item\emph{\href{https://developer.android.com/studio/test/espresso-test-recorder.html}{Espresso Test Recorder}}:
Designed and prototyped an Android Studio plugin that records user interactions
via instrumentation and synthesizes repeatable
\href{https://developer.android.com/training/testing/ui-testing/espresso-testing.html}{Espresso}
test code from the logs.
        \end{itemize}
        }
    \item
      \ressubheading{{Microsoft Research}}{Redmond, WA}{Research Intern}{May -- Aug 2014}
        {
\small
        \begin{itemize}
\item\emph{AppFormer: Synthesizing Cross-Platform Mappings from Examples}:
Researched automatic creation of platform-to-platform mappings
(e.g., Android to Windows Phone) by logging example apps' behaviors and
summarizing them via template-based program synthesis.
        \end{itemize}
        }
    \item
      \ressubheading{{University of Maryland, College Park}}{College Park, MD}{Teaching Assistant}{Aug 2010 -- May 2011}
        {
\small
        \begin{itemize}
\item\emph{CMSC 330 Organization of Programming Languages by Dr. Jeffrey S. Foster}:
Designed meta-scripts that generate testing scripts
that conduct regression tests over hundreds of students' projects.
\item\emph{CMSC 132 Object-Oriented Programming II by Larry Herman}:
Assisted in teaching basic algorithms and data structures in Java.
        \end{itemize}
        }

    \item
      \ressubheading{{Agency for Defense Development}}{Daejeon, Korea}{Researcher}{Feb 2007 -- Jul 2010}
        {
\small
        \begin{itemize}
\item Worked with an ELINT (ELectric INTelligence) team to build a pod-style ELINT system,
\item Researched an adaptive way to test the system with
a minimal number of input RF signals~\cite{valid09},
\item Designed a compact, hard-to-decompose data format for ELINT system missions, and
\item Developed a passive geolocation algorithm for RF signals.
        \end{itemize}
        }
%\bigskip % PAGE BREAK
    \item
      \ressubheading{{KAIST}}{Daejeon, Korea}{Research Assistant}{Mar 2006 -- Feb 2007}
        {
\small
        \begin{itemize}
\item\emph{\href{https://github.com/jsjeon/rtfa}{RTFA: Layout Transformation for Heap Objects}}~\cite{TACO09, cc07}:
Developed a compiler optimization that infers data structure access patterns
and transforms heap layouts to improve program performance
by increasing cache hit ratios
(\emph{won an Outstanding Master's Thesis Award from the department}).
\item\emph{Raccon: Buffer Overrun Analyzer for C Programs}~\cite{KIISE05, kcc06}:
Modified a buffer overrun analyzer for C programs so that
it can distinguish k different call contexts during analysis.
        \end{itemize}
        }
    \item
      \ressubheading{KAIST}{Daejeon, Korea}{Teaching Assistant}{Mar 2005 -- Feb 2006}
        {
\small
        \begin{itemize}
\item\emph{CS220 Programming Principles by Dr. Taisook Han}:
Assisted in teaching functional programming via Dr. Scheme, which is now Racket.
\item\emph{CS230 System Programming by Dr. Jinsoo Kim}:
Led a small group of students and helped them learn shell programming
(\emph{won an Excellent Teaching Assistant Award from the department}).
        \end{itemize}
        }
    \item
      \ressubheading{Tmax Soft}{Seongnam, Korea}{Internship}{Jul -- Aug 2004}
        {
\small
        \begin{itemize}
\item Programmed a simple bulletin board system (BBS) based on Java Servlet and JSP.
        \end{itemize}
        }
\end{itemize}  % End Experience list

%%%%%%%%%%%%%%%%%%%%%%

\resheading{Awards and Activities}
  \begin{description}
\item\textbf{Scholarships and Awards:}
\begin{itemize}
\item CAV15 Student Scholarship \hfill 2015
\item Google Fellowship Nominee, Dept. of Computer Science, UMD \hfill 2013 
\item Outstanding Master's Thesis Award, Dept. of Computer Science, KAIST \hfill 2007
\item Excellent Teaching Assistant Award, Dept. of Computer Science, KAIST \hfill 2005
\item \emph{Magna Cum Laude}, KAIST \hfill 2005
%\item Scholarship from Jungsoo Foundation \hfill 2003 -- 2004
\end{itemize}
\item\textbf{Professional Activities:}
\begin{itemize}
\item Reviewer, IEEE Transactions on Mobile Computing (TMC) \hfill 2015
\item Artifact Evaluation Committee, PLDI '15 \hfill 2015
\item Reviewer, POPL '15, ICSE '15 \hfill 2014
\item Reviewer, Journal of Information Security and Applications (JISA) \hfill 2014
\item Reviewer, IEEE Transactions on Dependable and Secure Computing (TDSC) \hfill 2013
\end{itemize}
\item\textbf{Extracurricular Activities:}
\begin{itemize}
\item Chair of Academic Affair, Washington Metro Chapter, Korean-American Scientist and Engineer's Association (KSEA) \hfill 2014 -- 2015
\item Staff of Korea Graduate Student Associates (KGSA), UMD \hfill 2013 -- 2015
\item Vice president of Korea Graduate Student Associates (KGSA), UMD \hfill 2011 -- 2013
\item Vice president of the senior class in Dept. of Computer Science, KAIST \hfill 2004
\end{itemize}
  \end{description} % End Skills list

\resheading{\href{http://scholar.google.com/citations?user=6l5Wb4QAAAAJ&hl=en\%22}{Publications}}
\begingroup
\renewcommand{\section}[2]{}
%
\begin{description}
\item\textbf{Journal Articles}
%
\nociteart{FMSD17}
\nociteart{TACO09}
\nociteart{KIISE05}
\bibliographystyleart{unsrt}
\bibliographyart{cv}
%
\item\textbf{Conference/Workshop Proceedings}
%
\nociteconf{icse16}
\nociteconf{esorics15}
\nociteconf{fse15}
\nociteconf{cav15}
\nociteconf{security14}
\nociteconf{spsm12}
\nociteconf{valid09}
\nociteconf{kcc07}
\nociteconf{cc07}
\nociteconf{kcc06}
\bibliographystyleconf{unsrt}
\bibliographyconf{cv}
%
\item\textbf{Posters}
%
\nocitepost{popl14}
\bibliographystylepost{unsrt}
\bibliographypost{cv}
%
\item\textbf{Technical Reports}
%
\nocitetr{jsk-tr}
\nocitetr{esorics15-tr}
\nocitetr{troyd}
\nocitetr{symdroid}
\nocitetr{drandroid}
\nocitetr{acplib}
\bibliographystyletr{unsrt}
\bibliographytr{cv}
%
\end{description}
%
\endgroup

\end{document}
