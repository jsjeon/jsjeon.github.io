%%%%%%%%%%%%%%%%%
% This is an sample CV template created using altacv.cls
% (v1.7.2, 28 Aug 2024) written by LianTze Lim (liantze@gmail.com), based on the
% CV created by BusinessInsider at http://www.businessinsider.my/a-sample-resume-for-marissa-mayer-2016-7/?r=US&IR=T
%
%% It may be distributed and/or modified under the
%% conditions of the LaTeX Project Public License, either version 1.3
%% of this license or (at your option) any later version.
%% The latest version of this license is in
%%    http://www.latex-project.org/lppl.txt
%% and version 1.3 or later is part of all distributions of LaTeX
%% version 2003/12/01 or later.
%%%%%%%%%%%%%%%%

%% Use the "normalphoto" option if you want a normal photo instead of cropped to a circle
% \documentclass[10pt,a4paper,normalphoto]{altacv}

\documentclass[9pt,letter,ragged2e,withhyper]{altacv}
%% AltaCV uses the fontawesome5 and simpleicons packages.
%% See http://texdoc.net/pkg/fontawesome5 and http://texdoc.net/pkg/simpleicons for full list of symbols.

% Change the page layout if you need to
\geometry{left=1.25cm,right=1.25cm,top=1.5cm,bottom=1.5cm,columnsep=1.2cm}

% The paracol package lets you typeset columns of text in parallel
\usepackage{paracol}


% Change the font if you want to, depending on whether
% you're using pdflatex or xelatex/lualatex
% WHEN COMPILING WITH XELATEX PLEASE USE
% xelatex -shell-escape -output-driver="xdvipdfmx -z 0" mmayer.tex
\ifxetexorluatex
  % If using xelatex or lualatex:
  \setmainfont{Lato}
\else
  % If using pdflatex:
  \usepackage[default]{lato}
\fi

% Change the colours if you want to
\definecolor{VividPurple}{HTML}{3E0097}
\definecolor{SlateGrey}{HTML}{2E2E2E}
\definecolor{LightGrey}{HTML}{666666}
\definecolor{DarkPastelRed}{HTML}{450808}
\definecolor{PastelRed}{HTML}{8F0D0D}
\definecolor{GoldenEarth}{HTML}{E7D192}
\colorlet{name}{black}
\colorlet{tagline}{PastelRed}
\colorlet{heading}{DarkPastelRed}
\colorlet{headingrule}{GoldenEarth}
\colorlet{subheading}{PastelRed}
\colorlet{accent}{PastelRed}
\colorlet{emphasis}{SlateGrey}
\colorlet{body}{LightGrey}

% Change some fonts, if necessary
% \renewcommand{\namefont}{\Huge\rmfamily\bfseries}
% \renewcommand{\personalinfofont}{\footnotesize}
% \renewcommand{\cvsectionfont}{\LARGE\rmfamily\bfseries}
% \renewcommand{\cvsubsectionfont}{\large\bfseries}

% Change the bullets for itemize and rating marker
% for \cvskill if you want to
\renewcommand{\cvItemMarker}{{\small\textbullet}}
\renewcommand{\cvRatingMarker}{\faCircle}
% ...and the markers for the date/location for \cvevent
% \renewcommand{\cvDateMarker}{\faCalendar*[regular]}
% \renewcommand{\cvLocationMarker}{\faMapMarker*}


% If your CV/résumé is in a language other than English,
% then you probably want to change these so that when you
% copy-paste from the PDF or run pdftotext, the location
% and date marker icons for \cvevent will paste as correct
% translations. For example Spanish:
% \renewcommand{\locationname}{Ubicación}
% \renewcommand{\datename}{Fecha}


%% Use (and optionally edit if necessary) this .tex if you
%% want to use an author-year reference style like APA(6)
%% for your publication list
% \input{pubs-authoryear.tex}

%% Use (and optionally edit if necessary) this .tex if you
%% want an originally numerical reference style like IEEE
%% for your publication list
% \input{pubs-num.tex}

%% sample.bib contains your publications
% \addbibresource{sample.bib}

\begin{document}
\name{Jinseong Jeon}
\tagline{Meta-program enthusiast: programs that input and/or output programs}
% Cropped to square from https://en.wikipedia.org/wiki/Marissa_Mayer#/media/File:Marissa_Mayer_May_2014_(cropped).jpg, CC-BY 2.0
%% You can add multiple photos on the left or right
% \photoR{2.5cm}{mmayer-wikipedia-cc-by-2_0}
% \photoL{2cm}{Yacht_High,Suitcase_High}
\personalinfo{%
  % Not all of these are required!
  % You can add your own with \printinfo{symbol}{detail}
  \email{jsjeon@cs.umd.edu}
%   \phone{000-00-0000}
  \mailaddress{425 Urban Plz}
  \location{Kirkland, WA}
  \homepage{jsjeon.github.io}
  % \twitter{@jsjeon}
  % \xtwitter{@jsjeon}
  \linkedin{jsjeon}
%   \github{github.com/jsjeon} % I'm just making this up though.
%   \orcid{0000-0000-0000-0000} % Obviously making this up too.
  %% You can add your own arbitrary detail with
  %% \printinfo{symbol}{detail}[optional hyperlink prefix]
  % \printinfo{\faPaw}{Hey ho!}
  %% Or you can declare your own field with
  %% \NewInfoFiled{fieldname}{symbol}[optional hyperlink prefix] and use it:
  % \NewInfoField{gitlab}{\faGitlab}[https://gitlab.com/]
  % \gitlab{your_id}
	%%
  %% For services and platforms like Mastodon where there isn't a
  %% straightforward relation between the user ID/nickname and the hyperlink,
  %% you can use \printinfo directly e.g.
  % \printinfo{\faMastodon}{@username@instace}[https://instance.url/@username]
  %% But if you absolutely want to create new dedicated info fields for
  %% such platforms, then use \NewInfoField* with a star:
  % \NewInfoField*{mastodon}{\faMastodon}
  %% then you can use \mastodon, with TWO arguments where the 2nd argument is
  %% the full hyperlink.
  % \mastodon{@username@instance}{https://instance.url/@username}
}

\makecvheader

%% Depending on your tastes, you may want to make fonts of itemize environments slightly smaller
\AtBeginEnvironment{itemize}{\small}

%% Set the left/right column width ratio to 6:4.
\columnratio{0.6}

% Start a 2-column paracol. Both the left and right columns will automatically
% break across pages if things get too long.
\begin{paracol}{2}

\cvsection{Experience}

\cvevent{Tech Lead / Staff Software Engineer}{Google}{Mar 2023 / Nov 2024 -- Present}{Kirkland, WA}

\begin{itemize}
\item\emph{Kotlin compiler at Google}:
Leading Kotlin frontend and JVM compiler efforts within Google
\begin{itemize}
\item Spearheaded K2 adoption in Google monorepo
\item Guided Android Studio migration to K2 and K2 IDE plugin
\item Drove tooling migrations: Android Lint and Metalava in AndroidX
\end{itemize}
\end{itemize}

\divider

\cvevent{Senior Software Engineer}{Google}{May 2019 -- Oct 2024}{Kirkland, WA}
\begin{itemize}
\item\emph{\href{https://github.com/JetBrains/intellij-community}{IntelliJ IDEA}} (\href{https://github.com/JetBrains/intellij-community/commits?author=jsjeon}{\underline{contributions}}):
Designed and built K2 UAST from scratch:
\begin{itemize}
\item Unified AST for Java and Kotlin, backbone of Android Lint
\item Landed to AndroidX: average 1.31x, up to 1.72x faster
\item Adopted by external companies: Meta, Square, and Mozilla
\end{itemize}
\item\emph{\href{https://github.com/JetBrains/kotlin}{Kotlin compiler}} (\href{https://github.com/JetBrains/kotlin/commits?author=jsjeon}{\underline{contributions}}):
Contributed to K2, the new Kotlin compiler frontend, end-to-end:
\begin{itemize}
\item Deserialization: to load Kotlin stdlib, other libraries or modules
\item Frontend IR generation (a.k.a. parsing)
\item Resolution: types, declarations, call targets, SAM/\texttt{suspend} conversion
\item Static analyses: control-flow / data-flow analysis, diagnostics
\item Conversion to backend IR
\item Serialization: to generate \texttt{@Metadata} for reflection
\end{itemize}
\end{itemize}

\divider

\cvevent{Software Engineer}{Google}{Feb 2016 -- Apr 2019}{Kirkland, WA}
\begin{itemize}
\item\emph{\href{https://developer.android.com/studio/preview/index.html}{Android Compiler Toolchain}}:
\href{https://r8.googlesource.com/r8}{D8 dexer and R8 shrinker} (\href{https://r8-review.googlesource.com/q/author:jsjeon+status:merged}{\underline{contributions}}):
Researched; designed; implemented; and deployed optimizations and obfuscations, such as:
\begin{itemize}
\item local type/nullability analysis, call-site optimization (e.g., remove Kotlin intrinsics calls)
\item \texttt{StringBuilder} optimization, compile-time reflection simplification, constant/call canonicalization
\item Kotlin \texttt{@Metadata} rewriting, identifier string obfuscation, package obfuscation
\end{itemize}
\item\emph{\href{https://cloud.google.com/compute/}{Google Compute Engine}}:
\href{https://cloudplatform.googleblog.com/2018/06/Introducing-sole-tenant-nodes-for-Google-Compute-Engine.html}{Sole-tenant nodes},
\href{https://cloudplatform.googleblog.com/2017/09/committed-use-discounts-for-Google-Compute-Engine-now-generally-available.html}{Committed use discounts}
\end{itemize}

\divider

\cvevent{Research Assistant}{University of Maryland}{Jun 2011 -- Feb 2016}{College Park, MD}

\begin{itemize}
\item\emph{\href{https://github.com/plum-umd/pasket}{\textsc{Pasket}}: Synthesizing Framework Models for Symbolic Execution}:
Researched and developed scalable synthesis of models for object-oriented, event-driven frameworks, such as Android
\item\emph{\href{https://github.com/plum-umd/redexer}{Redexer: Dalvik Bytecode Instrumentation Framework}}:
Developed a general-purpose bytecode rewriting framework for Android
\end{itemize}

\divider

\cvevent{Software Engineering Intern}{Google}{May -- Aug 2015}{Mountain View, CA}
\begin{itemize}
\item\emph{\href{https://developer.android.com/studio/test/espresso-test-recorder.html}{Espresso Test Recorder}}:
Designed and prototyped an Android Studio plugin that records user interactions
via instrumentation and synthesizes repeatable
\href{https://developer.android.com/training/testing/ui-testing/espresso-testing.html}{Espresso}
test code from the logs
\end{itemize}


%\cvsection{A Day of My Life}
%
%% Adapted from @Jake's answer from http://tex.stackexchange.com/a/82729/226
%% \wheelchart{outer radius}{inner radius}{
%% comma-separated list of value/text width/color/detail}
%% Some ad-hoc tweaking to adjust the labels so that they don't overlap
%\hspace*{-1em}  %% quick hack to move the wheelchart a bit left
%\wheelchart{1.5cm}{0.5cm}{%
%  10/13em/accent!30/Sleeping \& dreaming about work,
%  25/9em/accent!60/Public resolving issues with Yahoo!\ investors,
%  5/11em/accent!10/\footnotesize\\[1ex]New York \& San Francisco Ballet Jawbone board member,
%  20/11em/accent!40/Spending time with family,
%  5/8em/accent!20/\footnotesize Business development for Yahoo!\ after the Verizon acquisition,
%  30/9em/accent/Showing Yahoo!\ \mbox{employees} that their work has meaning,
%  5/8em/accent!20/Baking cupcakes
%}

% use ONLY \newpage if you want to force a page break for
% ONLY the currentc column
%\newpage
%
%\cvsection{Publications}
%
%% Specify your last name(s) and first name(s) as given in the .bib to automatically bold your own name in the publications list.
%% One caveat: You need to write \bibnamedelima where there's a space in your name for this to work properly; or write \bibnamedelimi if you use initials in the .bib
%% You can specify multiple names, especially if you have changed your name or if you need to highlight multiple authors.
%\mynames{Lim/Lian\bibnamedelima Tze,
%  Wong/Lian\bibnamedelima Tze,
%  Lim/Tracy,
%  Lim/L.\bibnamedelimi T.}
%% MAKE SURE THERE IS NO SPACE AFTER THE FINAL NAME IN YOUR \mynames LIST
%
%\nocite{*}

%\printbibliography[heading=pubtype,title={\printinfo{\faBook}{Books}},type=book]
%
%\divider
%
%\printbibliography[heading=pubtype,title={\printinfo{\faFile*[regular]}{Journal Articles}}, type=article]
%
%\divider
%
%\printbibliography[heading=pubtype,title={\printinfo{\faUsers}{Conference Proceedings}},type=inproceedings]

%% Switch to the right column. This will now automatically move to the second
%% page if the content is too long.
\switchcolumn

%\cvsection{Life Philosophy}
%\begin{quote}
%``If you don't have any shadows, you're not standing in the light.''
%\end{quote}

\cvsection{Proud of}

\cvachievement{\faTrophy}{5 Spot bonuses}{Kotlin compiler testing and UAST migration at Google scale}

\divider

\cvachievement{\faHeartbeat}{19 Peer bonuses}{Helping several partner teams and colleagues; fixing long-standing issues; beefing up tests; etc.}

\divider

\cvachievement{\faChartLine}{K2 adoption}{AGP 8.6+ and 1/3 Android Studio users while K2 IDE is not supported yet}

%\cvsection{Strengths}
%
%\cvtag{Hard-working (18/24)}
%\cvtag{Persuasive}\\
%\cvtag{Motivator \& Leader}
%
%\divider\smallskip
%
%\cvtag{UX}
%\cvtag{Mobile Devices \& Applications}
%\cvtag{Product Management \& Marketing}

\cvsection{Skills}

\begin{itemize}
\item Compiler, Compiler Optimization
\item Static Program Analysis
\item Program Synthesis
\item Automated Software Testing
\item JVM Application Performance Optimization
\end{itemize}

\cvsection{Programming Languages}

\begin{itemize}
\item Java, Kotlin: Professional Proficiency
\item Python, Ruby, OCaml: Working Proficiency
\item C\#, C++, C, Bash: Basic Working Knowledge
\end{itemize}

\cvsection{Languages}

%\cvskill{Korean}{5}
% \divider
%
%\cvskill{English}{4.5}
% \divider
%
%\cvskill{German}{3.5} %% supports X.5 values.

\begin{itemize}
\item Korean: Native
\item English: Professional Proficiency
\end{itemize}

\cvsection{Education}

\cvevent{Ph. D. \ in Computer Science}{University of Maryland, College Park}{Aug 2010 -- Feb 2016}{}

\divider\smallskip

\cvevent{M.S.\ in Computer Science}{KAIST}{Mar 2005 -- Feb 2007}{}

\divider\smallskip

\cvevent{B.S.\ in Computer Science}{KAIST}{Mar 2001 -- Feb 2005}{}

%\newpage
%
%\cvsection{Referees}
%
% \cvref{name}{email}{mailing address}
%\cvref{Prof.\ Alpha Beta}{Institute}{a.beta@university.edu}
%{Address Line 1\\Address line 2}
%
%\divider
%
%\cvref{Prof.\ Gamma Delta}{Institute}{g.delta@university.edu}
%{Address Line 1\\Address line 2}

\end{paracol}

\end{document}
